\documentclass{jsarticle}
\usepackage{moreverb}
\usepackage[dvipdfmx]{graphicx, hyperref}
\usepackage{float}
\usepackage{amsmath}
\usepackage{amssymb}

\usepackage{atbegshi}
\ifnum 42146=\euc"A4A2 \AtBeginShipoutFirst{\special{pdf:tounicode EUC-UCS2}}\else
\AtBeginShipoutFirst{\special{pdf:tounicode 90ms-RKSJ-UCS2}}\fi

\bibliographystyle{junsrt}

\title{「同期・引き込み現象について」課題}
\author{先進理工学研究科物理学及応用物理学専攻 山崎研究室 M1 藤本將太郎}
\date{\today}

\begin{document}
\maketitle

\section{はじめに}
この課題では、同期現象についての単純でありながら効果的な蔵本モデルのシミュレーションによる結果を通して、同期現象の起こるメカニズムの理解とともに、数値シミュレーションの手法を身に付けることを目的としています。

今回は導入の簡単さのために、表計算ソフト(エクセル等)を用いて計算を行うことを考えていますが、一般的にはこれらの数値シミュレーションはプログラミング言語を用いて行われることが多いと思われます。4年生で研究室に所属する際の配属研究室にもよりますが、数値シミュレーションを行う研究室は少なくありません。そういった意味で、今回プログラミング言語を使ってシミュレーションするかどうかにかかわりなく、プログラミングの授業や独学で勉強してみることをおすすめします。

また、先日配布された資料とは若干異なる内容になっているかと思いますので、よく確認してください。何か課題に関してや、そうでなくても疑問があれば、研究室(55N-302)に来て質問したり、学読で参考書を探したり、ネット上にある論文を読んで勉強するかしてみてください。今回シミュレーションしてもらうモデルは蔵本モデルと呼ばれる比較的単純なモデルですが、しかしながらこのモデルが導出されるには、実際には様々な要素を考える必要があります。これらの点についても発表できればいいのでは無いかと思います。

以下に課題としていくつかシミュレーションで調べてみてほしいことを記述していますが、個人的に興味がわいたことや、他の1年生に説明する上で付け加える必要があると思った事柄については、自分でも問題を設定して取り組んでみてください。

\section{課題}
\subsection{実験・観察}
    2つのメトロノームを用いて、実際に同期現象を確認せよ。どのような実験の設定にすると同期を見ることができるだろうか?同相同期、逆相同期はどのようなパラメータの違いによって引き起こされているものとみなせるだろうか?

\subsection{モデルにより得られる時系列の確認}
    同期現象をうまく説明する単純なモデルとして、蔵本モデルと呼ばれる、平均場を仮定し相互作用が位相差の正弦関数で表されるようなものが考えられている:
    $$\theta_{i} = \omega_{i}- \frac{K}{N}\sum_{i=1}^{N}\sin(\theta_{i}-\theta_{j}),\ \ \ i=1,2,\dots ,N$$  
    この蔵本モデルにおいて$N=2$として、$\omega_{i},K$に適当なパラメータと初期値$\theta_{1}(0), \theta_{2}(0), \dot{\theta_{1}}, \dot{\theta_{2}}$を定め、ひとつの場合について単純に時系列をプロットしてみよ($\theta$の範囲が$-\pi \le \theta \le \pi$となるようにすること)。\\
    \emph{ヒント:}  $\theta$と$\dot{\theta}$の大きさは、最終的な結果にはあまり寄与しないことがわかるので、適当に決めてよい。$\omega$は、例えば$\omega_{1} = 1, \omega_{2} = 2$として、$K=0.9\sim 1$の間を調べてみよ。時間ステップ$T$は$T=100$ほどとれば十分である。

\subsection{相空間上の振る舞い}
    上で試したいくつかの特徴的なパラメータについて、相空間における振る舞いについても調べよ。相空間とは力学系において位置と運動量を座標とする空間のことを表す。今の場合、横軸$\theta$、縦軸$\dot{\theta}$として散布図としてプロットせよ。この時の点の分布の形から何が分かるだろうか?

\subsection{オーダーパラメータ$R$と結合の強度$K$の間の関係}
    適当なパラメータを選ぶと、ある有限時間の後に2つの振動子は同期する。同期している状態と、そうでない状態を見分けるのに用いられる指標として、振動子の位相を円周上に考えた時のその重心の位置が挙げられる。2つの振動子が完全に同相同期している時には、この重心の円の中心からの距離$R$は円の半径に等しくなるはずである。円運動している円の半径を$1$として$K$を横軸、縦軸に$R$をとって、同期相転移が起こる臨界点$K_{c}$を概算せよ。\\
    この重心の半径は、相転移の文脈から、オーダーパラメーター(秩序変数)$r$とも呼ばれており、以下の式で定義されている:
    $$re^{i\phi} = \frac{1}{N}\sum_{j=1}^{N} e^{i\theta_{j}}$$
    ここで、$\phi$はすべての振動子の位相の平均であり、$r$は振動子のコヒーレンスを表している。2振動子の今の場合、上のオーダーパラメータをあらわに書くと
    $$r = e^{-i\left(\frac{1}{2}(\theta_{1} + \theta_{2})\right)} \frac{1}{2} (e^{i\theta_{1}} + e^{i\theta_{2}})$$
    のようになる。\\
    \emph{ヒント:} $\theta_{1}=0.1$, $\theta_{2}=0.5$, $\dot{\theta}=1.0$, $\dot{\theta_{2}} = 1.0$, $\omega_{1} = 1.0$, $\omega_{2} = 2.0$, $0\le K \le 2$の設定で計算を行い、$T=1000$までの計算のうち、最後の500個のデータを平均して$R$を求めてみよ。
\end{document}
