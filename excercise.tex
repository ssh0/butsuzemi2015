\documentclass{jsarticle}
\usepackage{moreverb}
\usepackage[dvipdfmx]{graphicx, hyperref}
\usepackage{float}
\usepackage{amsmath}
\usepackage{amssymb}

\usepackage{atbegshi}
\ifnum 42146=\euc"A4A2 \AtBeginShipoutFirst{\special{pdf:tounicode EUC-UCS2}}\else
\AtBeginShipoutFirst{\special{pdf:tounicode 90ms-RKSJ-UCS2}}\fi

\bibliographystyle{junsrt}

\title{「同期・引き込み現象について」課題}
\author{先進理工学研究科物理学及応用物理学専攻 山崎研究室 M1 藤本將太郎}
\date{\today}

\begin{document}
\maketitle

\section{はじめに}
この課題では、同期現象についての単純でありながら効果的な蔵本モデルのシミュレーションによる結果を通して、同期現象の起こるメカニズムの理解とともに、数値シミュレーションの手法を身に付けることを目的としています。

今回は導入の簡単さのために、表計算ソフト(エクセル等)を用いて計算を行うことを考えていますが、一般的にはこれらの数値シミュレーションはプログラミング言語を用いて行われることが多いと思われます。4年生で研究室に所属する際の配属研究室にもよりますが、数値シミュレーションを行う研究室は少なくありません。そういった意味で、今回プログラミング言語を使ってシミュレーションするかどうかにかかわりなく、プログラミングの授業や独学で勉強してみることをおすすめします。

また、先日配布された資料とは若干異なる内容になっているかと思いますので、よく確認してください。何か課題に関してや、そうでなくても疑問があれば、研究室(55N-302)に来て質問したり、学読で参考書を探したり、ネット上にある論文を読んで勉強するかしてみてください。今回シミュレーションしてもらうモデルは蔵本モデルと呼ばれる比較的単純なモデルですが、しかしながらこのモデルが導出されるには、実際には様々な要素を考えて置く必要があります。これらの点についてもすこし考えてもらいたいと思います。

以下に課題としていくつかシミュレーションで調べてみてほしいことを記述していますが、個人的に興味がわいたことや、他の1年生に説明する上で付け加える必要があると思った事柄については、自分でも問題を設定して取り組んでみてください。

\section{課題}
\begin{itemize}
    \item 
\end{itemize}
\end{document}
